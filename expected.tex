% Options for packages loaded elsewhere
\PassOptionsToPackage{unicode}{hyperref}
\PassOptionsToPackage{hyphens}{url}
%
\documentclass[
]{article}
\usepackage{amsmath,amssymb}
\usepackage{lmodern}
\usepackage{iftex}
\ifPDFTeX
  \usepackage[T1]{fontenc}
  \usepackage[utf8]{inputenc}
  \usepackage{textcomp} % provide euro and other symbols
\else % if luatex or xetex
  \usepackage{unicode-math}
  \defaultfontfeatures{Scale=MatchLowercase}
  \defaultfontfeatures[\rmfamily]{Ligatures=TeX,Scale=1}
\fi
% Use upquote if available, for straight quotes in verbatim environments
\IfFileExists{upquote.sty}{\usepackage{upquote}}{}
\IfFileExists{microtype.sty}{% use microtype if available
  \usepackage[]{microtype}
  \UseMicrotypeSet[protrusion]{basicmath} % disable protrusion for tt fonts
}{}
\makeatletter
\@ifundefined{KOMAClassName}{% if non-KOMA class
  \IfFileExists{parskip.sty}{%
    \usepackage{parskip}
  }{% else
    \setlength{\parindent}{0pt}
    \setlength{\parskip}{6pt plus 2pt minus 1pt}}
}{% if KOMA class
  \KOMAoptions{parskip=half}}
\makeatother
\usepackage{xcolor}
\usepackage{color}
\usepackage{fancyvrb}
\newcommand{\VerbBar}{|}
\newcommand{\VERB}{\Verb[commandchars=\\\{\}]}
\DefineVerbatimEnvironment{Highlighting}{Verbatim}{commandchars=\\\{\}}
% Add ',fontsize=\small' for more characters per line
\newenvironment{Shaded}{}{}
\newcommand{\AlertTok}[1]{\textcolor[rgb]{1.00,0.00,0.00}{\textbf{#1}}}
\newcommand{\AnnotationTok}[1]{\textcolor[rgb]{0.38,0.63,0.69}{\textbf{\textit{#1}}}}
\newcommand{\AttributeTok}[1]{\textcolor[rgb]{0.49,0.56,0.16}{#1}}
\newcommand{\BaseNTok}[1]{\textcolor[rgb]{0.25,0.63,0.44}{#1}}
\newcommand{\BuiltInTok}[1]{\textcolor[rgb]{0.00,0.50,0.00}{#1}}
\newcommand{\CharTok}[1]{\textcolor[rgb]{0.25,0.44,0.63}{#1}}
\newcommand{\CommentTok}[1]{\textcolor[rgb]{0.38,0.63,0.69}{\textit{#1}}}
\newcommand{\CommentVarTok}[1]{\textcolor[rgb]{0.38,0.63,0.69}{\textbf{\textit{#1}}}}
\newcommand{\ConstantTok}[1]{\textcolor[rgb]{0.53,0.00,0.00}{#1}}
\newcommand{\ControlFlowTok}[1]{\textcolor[rgb]{0.00,0.44,0.13}{\textbf{#1}}}
\newcommand{\DataTypeTok}[1]{\textcolor[rgb]{0.56,0.13,0.00}{#1}}
\newcommand{\DecValTok}[1]{\textcolor[rgb]{0.25,0.63,0.44}{#1}}
\newcommand{\DocumentationTok}[1]{\textcolor[rgb]{0.73,0.13,0.13}{\textit{#1}}}
\newcommand{\ErrorTok}[1]{\textcolor[rgb]{1.00,0.00,0.00}{\textbf{#1}}}
\newcommand{\ExtensionTok}[1]{#1}
\newcommand{\FloatTok}[1]{\textcolor[rgb]{0.25,0.63,0.44}{#1}}
\newcommand{\FunctionTok}[1]{\textcolor[rgb]{0.02,0.16,0.49}{#1}}
\newcommand{\ImportTok}[1]{\textcolor[rgb]{0.00,0.50,0.00}{\textbf{#1}}}
\newcommand{\InformationTok}[1]{\textcolor[rgb]{0.38,0.63,0.69}{\textbf{\textit{#1}}}}
\newcommand{\KeywordTok}[1]{\textcolor[rgb]{0.00,0.44,0.13}{\textbf{#1}}}
\newcommand{\NormalTok}[1]{#1}
\newcommand{\OperatorTok}[1]{\textcolor[rgb]{0.40,0.40,0.40}{#1}}
\newcommand{\OtherTok}[1]{\textcolor[rgb]{0.00,0.44,0.13}{#1}}
\newcommand{\PreprocessorTok}[1]{\textcolor[rgb]{0.74,0.48,0.00}{#1}}
\newcommand{\RegionMarkerTok}[1]{#1}
\newcommand{\SpecialCharTok}[1]{\textcolor[rgb]{0.25,0.44,0.63}{#1}}
\newcommand{\SpecialStringTok}[1]{\textcolor[rgb]{0.73,0.40,0.53}{#1}}
\newcommand{\StringTok}[1]{\textcolor[rgb]{0.25,0.44,0.63}{#1}}
\newcommand{\VariableTok}[1]{\textcolor[rgb]{0.10,0.09,0.49}{#1}}
\newcommand{\VerbatimStringTok}[1]{\textcolor[rgb]{0.25,0.44,0.63}{#1}}
\newcommand{\WarningTok}[1]{\textcolor[rgb]{0.38,0.63,0.69}{\textbf{\textit{#1}}}}
\setlength{\emergencystretch}{3em} % prevent overfull lines
\providecommand{\tightlist}{%
  \setlength{\itemsep}{0pt}\setlength{\parskip}{0pt}}
\setcounter{secnumdepth}{-\maxdimen} % remove section numbering
\usepackage{multicol}
\ifLuaTeX
  \usepackage{selnolig}  % disable illegal ligatures
\fi
\IfFileExists{bookmark.sty}{\usepackage{bookmark}}{\usepackage{hyperref}}
\IfFileExists{xurl.sty}{\usepackage{xurl}}{} % add URL line breaks if available
\urlstyle{same} % disable monospaced font for URLs
\hypersetup{
  pdftitle={Columns - Multiple columns support in Pandoc's markdown},
  pdfauthor={Julien Dutant},
  hidelinks,
  pdfcreator={LaTeX via pandoc}}

\title{Columns - Multiple columns support in Pandoc's markdown}
\author{Julien Dutant}
\date{}

\begin{document}
\maketitle

\hypertarget{columns}{%
\section{Columns}\label{columns}}

Multiple columns support in Pandoc's markdown.

v1.1.2. Copyright: © 2021 Julien Dutant
\href{mailto:julien.dutant@kcl.ac.uk}{\nolinkurl{julien.dutant@kcl.ac.uk}}
License: MIT - see LICENSE file for details.

\hypertarget{introduction}{%
\subsection{Introduction}\label{introduction}}

This Lua filter for Pandoc provides multicolumn support in
\href{https://pandoc.org/}{Pandoc}'s markdown for outputs in
\texttt{html} and LaTeX/PDF. It supports several markdown syntaxes,
explicit column breaks, spanning elements, customisation and nesting.
Html output relies on \href{https://drafts.csswg.org/css-multicol}{CSS
Multi-column layout} and LaTeX/PDF outputs on the
\href{https://www.ctan.org/pkg/multicol}{\texttt{multicol} LaTeX
package}.

Limitations: in \texttt{html} output, support is limited to recent
browsers and variable across browsers.

This document also serves as a test document. To see the multi-columns
layouts of this document in action, you need to process it with
\texttt{pandoc} using this filter.

\hypertarget{pre-requistes}{%
\subsection{Pre-requistes}\label{pre-requistes}}

Requires \href{https://pandoc.org/}{Pandoc}. Copy the file
\texttt{columns.lua} in your working folder or in Pandoc's
\texttt{filter} folder. Called from the command line with a \texttt{-L}
or \texttt{-\/-lua-filter} option:

\begin{Shaded}
\begin{Highlighting}[]
\ExtensionTok{pandoc} \AttributeTok{{-}{-}lua{-}filter}\NormalTok{ columns.lua SOURCE.md }\AttributeTok{{-}o}\NormalTok{ DESTINATION.html}

\ExtensionTok{pandoc} \AttributeTok{{-}L}\NormalTok{ columns.lua SOURCE.md }\AttributeTok{{-}o}\NormalTok{ DESTINATION.pdf}
\end{Highlighting}
\end{Shaded}

Or from a \texttt{filters} field in a Pandoc defaults file. See the
\href{https://pandoc.org/MANUAL.html}{Pandoc documentation} for further
details.

For instance, to process the present documentation use:

\begin{Shaded}
\begin{Highlighting}[]
\ExtensionTok{pandoc} \AttributeTok{{-}L}\NormalTok{ columns.lua README.md }\AttributeTok{{-}o}\NormalTok{ readme.html}

\ExtensionTok{pandoc} \AttributeTok{{-}L}\NormalTok{ columns.lua README.md }\AttributeTok{{-}o}\NormalTok{ readme.pdf}
\end{Highlighting}
\end{Shaded}

\hypertarget{basic-usage}{%
\subsection{Basic usage}\label{basic-usage}}

\hypertarget{columns-1}{%
\subsubsection{Columns}\label{columns-1}}

In Pandoc markdown source specify a multicolumn section as follows:

\begin{Shaded}
\begin{Highlighting}[]
\NormalTok{::: columns}

\NormalTok{...content that will be spread over several columns...}

\NormalTok{:::}
\end{Highlighting}
\end{Shaded}

The filter will render this section as a multicolumns layout in
\texttt{html} and LaTeX, as illustrated below (you need to process this
document with \texttt{pandoc} using this filter to see the results in
\texttt{html} or \texttt{pdf}:

{\raggedcolumns\begin{multicols}{2}

Lorem ipsum dolor sit amet, consectetur adipiscing elit. Donec a ante in
mi ornare volutpat sed sit amet diam. Nullam interdum erat a augue
faucibus, nec tempus tortor sagittis. Aenean imperdiet imperdiet
dignissim. Nam aliquam blandit ex, sed molestie nibh feugiat ac. Morbi
feugiat convallis semper. Ut et consequat purus. Fusce convallis
vehicula enim in vulputate. Curabitur a augue arcu. Mauris laoreet
lectus arcu, sed elementum turpis scelerisque id. Etiam porta turpis
quis ipsum dictum vulputate. In ut convallis urna, at imperdiet nunc.
Cras laoreet, massa lobortis gravida egestas, lacus est pellentesque
arcu, imperdiet efficitur nibh dolor vel sapien. Sed accumsan
condimentum diam non pellentesque.

Vestibulum cursus nisi risus, sit amet consectetur massa suscipit nec.
Sed condimentum, est id iaculis ornare, purus risus finibus felis,
posuere congue est nibh eget dui. Maecenas orci erat, commodo auctor
justo quis, vestibulum mollis ex. Vivamus sed bibendum turpis. Donec
auctor, leo a cursus efficitur, quam urna dignissim enim, viverra
condimentum orci est non sem. Donec ac viverra nisl. Suspendisse ac
auctor massa. Mauris porttitor purus vel velit vehicula, sed efficitur
odio lacinia. Fusce sed odio arcu. Ut rhoncus lacus vel magna interdum
tincidunt. Nunc imperdiet finibus tincidunt.

\end{multicols}
}

This syntax is based on the
\href{https://pandoc.org/MANUAL.html\#divs-and-spans}{\texttt{fenced\_div}
syntax of Pandoc's' markdown}. At least three consecutive colons are
needed, both at the beginning and at then end of your multi-column
section (even if it runs until the end of your document). But more than
three are fine:

\begin{Shaded}
\begin{Highlighting}[]
\NormalTok{::: columns ::::::}

\NormalTok{...content that will be spread over several columns...}

\NormalTok{::::::::::::::}
\end{Highlighting}
\end{Shaded}

Each opening series of colons needs to be matched with a closing ones.
For readibility we usually match their number of colons but it's not
necessary (as the above illustrates). If you enclose sections within
sections (see container syntax, nesting, column spans and column breaks
below) you need to make sure that each opening series of colons is
matched by a closing one, otherwise Pandoc will not recognize them or
interpret them incorrectly.

Here \texttt{columns} is a \emph{attribute} of the fenced div (section).
As we'll see below, these sections can have more than a single
attribute. When they have several, they need to be specified within
curly brackets and \texttt{columns} should be preceded by a dot, as in:

\begin{Shaded}
\begin{Highlighting}[]
\NormalTok{::::: \{.columns .someattribute property=value\}}

\NormalTok{...content that will be spread over several columns...}

\NormalTok{:::::}
\end{Highlighting}
\end{Shaded}

\hypertarget{specifying-the-number-of-columns}{%
\subsubsection{Specifying the number of
columns}\label{specifying-the-number-of-columns}}

By default two columns are provided. You can specify the desired number
of columns in various ways:

\begin{Shaded}
\begin{Highlighting}[]
\NormalTok{::: twocolumns}

\NormalTok{::: three{-}columns}

\NormalTok{::: five\_columns}

\NormalTok{::: \{.columns column{-}count=3\}}
\end{Highlighting}
\end{Shaded}

Note that in \texttt{html} browsers may override your specified number
of columns.

\hypertarget{ragged-columns-latex-output-only}{%
\subsubsection{Ragged columns (LaTeX output
only)}\label{ragged-columns-latex-output-only}}

Default LaTeX/PDF output justifies columns vertically. That is, if
columns are explicitly broken at certain points, LaTeX ensures that the
text in each column occupies its full height by stretching
inter-paragraph space. In HTML output columns are always ``ragged'',
that is, inter-paragraph space isn't stretched and shorter columns have
blank space at the end.

If you want ragged columns in LaTeX, you can set this globally in the
document's metadata or on locally on a give \texttt{columns} Div. In the
document data, either of these keys will work:

\begin{verbatim}
ragged-columns: true
raggedcolumns: true
\end{verbatim}

Locally, add the \texttt{ragged} (or \texttt{raggedcolumns} or
\texttt{ragged-columns}) class to a \texttt{columns} Div:

\begin{Shaded}
\begin{Highlighting}[]
\NormalTok{::::: \{.columns .ragged\}}

\NormalTok{...}

\NormalTok{:::::}
\end{Highlighting}
\end{Shaded}

Note that this doesn't work on individual \texttt{column} Divs, only on
the \texttt{columns} Div that contains them.

There is a corresponding \texttt{justifiedcolumns} (alias
\texttt{justified-columns}) global setting and a \texttt{justified}
(alias \texttt{justifiedcolumns}, \texttt{justified-columns}) class for
specific \texttt{columns} Div.

{\begin{multicols}{3}

This column

is vertically short.

\columnbreak

This column

is vertically short.

\columnbreak

Lorem ipsum dolor sit amet, consectetur adipiscing elit. Donec a ante in
mi ornare volutpat sed sit amet diam. Nullam interdum erat a augue
faucibus, nec tempus tortor sagittis. Aenean imperdiet imperdiet
dignissim. Nam aliquam blandit ex, sed molestie nibh feugiat ac. Morbi
feugiat convallis semper. Ut et consequat purus. Fusce convallis
vehicula enim in vulputate.

\end{multicols}
}

Now in ragged columns mode:

{\raggedcolumns\begin{multicols}{3}

This column

is vertically short.

\columnbreak

This column

is vertically short.

\columnbreak

Lorem ipsum dolor sit amet, consectetur adipiscing elit. Donec a ante in
mi ornare volutpat sed sit amet diam. Nullam interdum erat a augue
faucibus, nec tempus tortor sagittis. Aenean imperdiet imperdiet
dignissim. Nam aliquam blandit ex, sed molestie nibh feugiat ac. Morbi
feugiat convallis semper. Ut et consequat purus. Fusce convallis
vehicula enim in vulputate.

\end{multicols}
}

\hypertarget{customizing-the-gap-and-rule-between-columns}{%
\subsubsection{Customizing the gap and rule between
columns}\label{customizing-the-gap-and-rule-between-columns}}

The gap and rule between columns can be customized too. The gap is
specified with a \texttt{columngap} (or \texttt{column-gap} or
\texttt{columnsep} or \texttt{column-sep}) attribute. The rule is
specified with a \texttt{column-rule} (or \texttt{columnrule}) attribute
using CSS syntax.

\begin{Shaded}
\begin{Highlighting}[]
\NormalTok{::: \{.columns columngap=3em column{-}rule="1px solid black"\}}

\NormalTok{::: \{.threecolumns columngap=4em column{-}rule="3pt solid blue"\}}
\end{Highlighting}
\end{Shaded}

Here is an illustration:

{\raggedcolumns\setlength{\columnsep}{4em}
\setlength{\columnseprule}{ 3pt}
\renewcommand{\columnseprulecolor}{\color{blue}}
\begin{multicols}{3}

Lorem ipsum dolor sit amet, consectetur adipiscing elit. Donec a ante in
mi ornare volutpat sed sit amet diam. Nullam interdum erat a augue
faucibus, nec tempus tortor sagittis. Aenean imperdiet imperdiet
dignissim. Nam aliquam blandit ex, sed molestie nibh feugiat ac. Morbi
feugiat convallis semper. Ut et consequat purus. Fusce convallis
vehicula enim in vulputate. Curabitur a augue arcu. Mauris laoreet
lectus arcu, sed elementum turpis scelerisque id. Etiam porta turpis
quis ipsum dictum vulputate. In ut convallis urna, at imperdiet nunc.
Cras laoreet, massa lobortis gravida egestas, lacus est pellentesque
arcu, imperdiet efficitur nibh dolor vel sapien. Sed accumsan
condimentum diam non pellentesque.

Vestibulum cursus nisi risus, sit amet consectetur massa suscipit nec.
Sed condimentum, est id iaculis ornare, purus risus finibus felis,
posuere congue est nibh eget dui. Maecenas orci erat, commodo auctor
justo quis, vestibulum mollis ex. Vivamus sed bibendum turpis. Donec
auctor, leo a cursus efficitur, quam urna dignissim enim, viverra
condimentum orci est non sem. Donec ac viverra nisl. Suspendisse ac
auctor massa. Mauris porttitor purus vel velit vehicula, sed efficitur
odio lacinia. Fusce sed odio arcu. Ut rhoncus lacus vel magna interdum
tincidunt. Nunc imperdiet finibus tincidunt.

\end{multicols}
}

\hypertarget{spanning-elements}{%
\subsubsection{Spanning elements}\label{spanning-elements}}

Elements that span across all columns are introduced as
\texttt{column-span} (or \texttt{columnspan}) sections:

\begin{Shaded}
\begin{Highlighting}[]
\NormalTok{::: columns ::::::::}

\NormalTok{content in columns}

\NormalTok{::::: column{-}span}

\NormalTok{\# This heading spans across all columns}

\NormalTok{:::::}

\NormalTok{content in columns}

\NormalTok{:::}
\end{Highlighting}
\end{Shaded}

Here is an illustration:

{\raggedcolumns\setlength{\columnseprule}{ 1pt}
\renewcommand{\columnseprulecolor}{\color{black}}
\begin{multicols}{2}

Lorem ipsum dolor sit amet, consectetur adipiscing elit. Donec a ante in
mi ornare volutpat sed sit amet diam. Nullam interdum erat a augue
faucibus, nec tempus tortor sagittis. Aenean imperdiet imperdiet
dignissim. Nam aliquam blandit ex, sed molestie nibh feugiat ac. Morbi
feugiat convallis semper. Ut et consequat purus. Fusce convallis
vehicula enim in vulputate. Curabitur a augue arcu. Mauris laoreet
lectus arcu, sed elementum turpis scelerisque id. Etiam porta turpis
quis ipsum dictum vulputate. In ut convallis urna, at imperdiet nunc.
Cras laoreet, massa lobortis gravida egestas, lacus est pellentesque
arcu, imperdiet efficitur nibh dolor vel sapien. Sed accumsan
condimentum diam non pellentesque.

\end{multicols}
\begin{multicols}{2}[\hypertarget{vestibulum-cursus-nisi-risus-sit-amet-consectetur-massa-suscipit-nec}{%
\section{Vestibulum cursus nisi risus, sit amet consectetur massa
suscipit
nec}\label{vestibulum-cursus-nisi-risus-sit-amet-consectetur-massa-suscipit-nec}}]

Sed condimentum, est id iaculis ornare, purus risus finibus felis,
posuere congue est nibh eget dui. Maecenas orci erat, commodo auctor
justo quis, vestibulum mollis ex. Vivamus sed bibendum turpis. Donec
auctor, leo a cursus efficitur, quam urna dignissim enim, viverra
condimentum orci est non sem. Donec ac viverra nisl. Suspendisse ac
auctor massa. Mauris porttitor purus vel velit vehicula, sed efficitur
odio lacinia. Fusce sed odio arcu. Ut rhoncus lacus vel magna interdum
tincidunt. Nunc imperdiet finibus tincidunt.

\end{multicols}
}

\hypertarget{explicitly-specifying-column-breaks}{%
\subsubsection{Explicitly specifying column
breaks}\label{explicitly-specifying-column-breaks}}

Column breaks can be explicitly specified. This can be done using
\texttt{\textbackslash{}columnbreak} or a \texttt{columnbreak} (or
\texttt{column-break}) section.

\begin{Shaded}
\begin{Highlighting}[]
\NormalTok{::: columns}

\NormalTok{This content is in a first column.}

\NormalTok{\textbackslash{}columnbreak}

\NormalTok{This content is in a second column.}

\NormalTok{:::: columnbreak}
\NormalTok{::::}

\NormalTok{This content is in a third column.}

\NormalTok{:::: column{-}break}
\NormalTok{::::}

\NormalTok{This content is in a fourth column.}

\NormalTok{:::}
\end{Highlighting}
\end{Shaded}

The result is:

{\raggedcolumns\begin{multicols}{4}

This content is in a first column.

\columnbreak

This content is in a second column.

\columnbreak

This content is in a third column.

\columnbreak

This content is in a fourth column.

\end{multicols}
}

\textbf{Warning and limitations}

\begin{itemize}
\tightlist
\item
  In \texttt{html}, browsers may ignore explicit column breaks.
\item
  A \texttt{\textbackslash{}columnbreak} break must be preceded by an
  empty line and occupy a line on its own.
\item
  A \texttt{:::\ columnbreak} break must be followed by a closing line
  of \texttt{:::}.
\end{itemize}

When columnbreaks are explicitly specified, they are used to determine
the number of columns. If the section both speficies a number of columns
and includes explicit columnbreaks, the greatest number is used.

\hypertarget{container-syntax}{%
\subsubsection{Container syntax}\label{container-syntax}}

A multicolumn section with explicit breaks can also be written using a
container syntax, with \texttt{column} sections included in a
\texttt{columns} section, as follows.

\begin{Shaded}
\begin{Highlighting}[]
\NormalTok{:::::: columns}

\NormalTok{::: column}

\NormalTok{First column content here}

\NormalTok{:::}

\NormalTok{::: column}

\NormalTok{Second column content}

\NormalTok{:::}

\NormalTok{:::::}
\end{Highlighting}
\end{Shaded}

This follows \href{https://pandoc.org/MANUAL.html\#columns}{Pandoc's
markdown syntax for \texttt{beamer} output}. Note that individual column
widths and further column attributes available in \texttt{beamer}
outputs are not supported here.

Container syntax and columnbreak syntax can be mixed, as in the example
below:

{\raggedcolumns\begin{multicols}{3}

Lorem ipsum dolor sit amet, consectetur adipiscing elit. Donec a ante in
mi ornare volutpat sed sit amet diam. Nullam interdum erat a augue
faucibus, nec tempus tortor sagittis. Aenean imperdiet imperdiet
dignissim. Nam aliquam blandit ex, sed molestie nibh feugiat ac. Morbi
feugiat convallis semper. Ut et consequat purus. Fusce convallis
vehicula enim in vulputate. Curabitur a augue arcu.

\columnbreak

Mauris laoreet lectus arcu, sed elementum turpis scelerisque id. Etiam
porta turpis quis ipsum dictum vulputate. In ut convallis urna, at
imperdiet nunc. Cras laoreet, massa lobortis gravida egestas, lacus est
pellentesque arcu, imperdiet efficitur nibh dolor vel sapien. Sed
accumsan condimentum diam non pellentesque.

\columnbreak

Vestibulum cursus nisi risus, sit amet consectetur massa suscipit nec.
Sed condimentum, est id iaculis ornare, purus risus finibus felis,
posuere congue est nibh eget dui. Maecenas orci erat, commodo auctor
justo quis, vestibulum mollis ex.

\end{multicols}
}

\hypertarget{advanced-usage}{%
\subsection{Advanced usage}\label{advanced-usage}}

\hypertarget{nesting}{%
\subsubsection{Nesting}\label{nesting}}

Multicolumn sections can be nested. Support for nesting may vary across
browsers. Here is an illustration:

{\raggedcolumns\begin{multicols}{3}

Lorem ipsum dolor sit amet, consectetur adipiscing elit. Donec a ante in
mi ornare volutpat sed sit amet diam. Nullam interdum erat a augue
faucibus, nec tempus tortor sagittis. Aenean imperdiet imperdiet
dignissim. Nam aliquam blandit ex, sed molestie nibh feugiat ac. Morbi
feugiat convallis semper. Ut et consequat purus. Fusce convallis
vehicula enim in vulputate. Curabitur a augue arcu.

\columnbreak

Mauris laoreet lectus arcu, sed elementum turpis scelerisque id. Etiam
porta turpis quis ipsum dictum vulputate. In ut convallis urna, at
imperdiet nunc.

{\raggedcolumns\begin{multicols}{2}

This is a two-column section nested within the middle column of a
three-column section.

\end{multicols}
}

Cras laoreet, massa lobortis gravida egestas, lacus est pellentesque
arcu, imperdiet efficitur nibh dolor vel sapien. Sed accumsan
condimentum diam non pellentesque.

\columnbreak

Vestibulum cursus nisi risus, sit amet consectetur massa suscipit nec.
Sed condimentum, est id iaculis ornare, purus risus finibus felis,
posuere congue est nibh eget dui. Maecenas orci erat, commodo auctor
justo quis, vestibulum mollis ex.

\end{multicols}
}

\hypertarget{number-of-columns}{%
\subsubsection{Number of columns}\label{number-of-columns}}

Number of columns can be specified in English up to ten. Accepted
patterns are \texttt{\textless{}number\textgreater{}columns},
\texttt{\textless{}number\textgreater{}-columns} and
\texttt{\textless{}number\textgreater{}\_columns}. Note that this is a
``class'', and should be preceded by a dot when specified along other
attributes within curly brackets:

\begin{verbatim}
::: twocolumns

::: {.three-columns columnsep=2em}

:::
\end{verbatim}

Alternatively, the \texttt{column-count} can be used to specify any
number of columns.

\begin{verbatim}
::: {.columns column-count=3}
\end{verbatim}

If both English names and \texttt{column-count} are used, the former
prevails.

\hypertarget{html-output}{%
\subsubsection{HTML output}\label{html-output}}

The \texttt{html} output looks like this. Without column breaks:

\begin{Shaded}
\begin{Highlighting}[]
\KeywordTok{\textless{}div} \ErrorTok{class}\OtherTok{=}\StringTok{"columns"} \ErrorTok{style}\OtherTok{=}\StringTok{"column{-}count: 2; column{-}rule: 1px solid black;"}\KeywordTok{\textgreater{}}

\NormalTok{Content that distributed in columns...}

\KeywordTok{\textless{}div} \ErrorTok{class}\OtherTok{=}\StringTok{"column{-}span"} \ErrorTok{style}\OtherTok{=}\StringTok{";"}\KeywordTok{\textgreater{}}
\NormalTok{Content that spreads across all columns}
\KeywordTok{\textless{}/div\textgreater{}}

\NormalTok{More content distributed in columns...}

\KeywordTok{\textless{}/div\textgreater{}}
\end{Highlighting}
\end{Shaded}

With columnbreaks:

\begin{Shaded}
\begin{Highlighting}[]
\KeywordTok{\textless{}div} \ErrorTok{class}\OtherTok{=}\StringTok{"columns"} \ErrorTok{style}\OtherTok{=}\StringTok{"column{-}count: 2;"}\KeywordTok{\textgreater{}}

\NormalTok{Content of the first column.}

\KeywordTok{\textless{}div} \ErrorTok{style}\OtherTok{=}\StringTok{"break{-}after: column;"}\KeywordTok{\textgreater{}\textless{}/div\textgreater{}}

\NormalTok{Content of the second column.}

\KeywordTok{\textless{}/div\textgreater{}}
\end{Highlighting}
\end{Shaded}

In CSS \texttt{break-after:\ column} means ``after this element, place a
column break''.

The classes \texttt{columns} and \texttt{column-span} are needed to
ensure that the first element of a multiple columns \texttt{div}, or the
first element after an element spanning across columns, have no top
margin. If they had we would get unwanted space at the beginning of the
first column. Thus the filter adds the following to the header:

\begin{Shaded}
\begin{Highlighting}[]
  \KeywordTok{\textless{}style\textgreater{}}
    \FunctionTok{.columns} \InformationTok{:first{-}child}\NormalTok{ \{}\KeywordTok{margin{-}top}\NormalTok{: }\DecValTok{0}\OperatorTok{;}\NormalTok{\}}
    \FunctionTok{.column{-}span} \OperatorTok{+} \OperatorTok{*}\NormalTok{ \{}\KeywordTok{margin{-}top}\NormalTok{: }\DecValTok{0}\OperatorTok{;}\NormalTok{\}}
  \KeywordTok{\textless{}/style\textgreater{}}
\end{Highlighting}
\end{Shaded}

\hypertarget{latex-output}{%
\subsubsection{LaTeX output}\label{latex-output}}

The LaTeX output looks as follows. Preamble:

\begin{Shaded}
\begin{Highlighting}[]
\BuiltInTok{\textbackslash{}usepackage}\NormalTok{\{}\ExtensionTok{multicol}\NormalTok{\}}
\end{Highlighting}
\end{Shaded}

Document body:

\begin{verbatim}
{\begin{multicols}{2}

content distributed over two columns

\end{multicols}
}
\end{verbatim}

With properties and explicit column breaks:

\begin{verbatim}
{\setlength{\columnsep}{4em}
\setlength{\columnseprule}{ 3pt}
\renewcommand{\columnseprulecolor}{\color{blue}}
\begin{multicols}{3}

content distributed over three columns

\end{multicols}
}
\end{verbatim}

Note that the \texttt{multicols} environment is wrapped within
\texttt{\{...\}}. This is to ensure that settings of
\texttt{\textbackslash{}columnsep},
\texttt{\textbackslash{}columnseprule} and
\texttt{\textbackslash{}columnseprulecolor} do not affect subsequent
\texttt{multicol} environments.

\hypertarget{contributing}{%
\subsection{Contributing}\label{contributing}}

Issues and pull requests are welcome. They can be submitted
\href{https://github.com/jdutant/columns}{to the repository}.

\hypertarget{references}{%
\subsection{References}\label{references}}

\begin{itemize}
\tightlist
\item
  \texttt{html}: \href{https://drafts.csswg.org/css-multicol}{CSS
  Multi-column layout}
\item
  LaTeX: \href{https://www.ctan.org/pkg/multicol}{\texttt{multicol}
  LaTeX package}
\item
  Pandoc: https://pandoc.org/lua-filters.html
\item
  Pandoc lua filters: https://pandoc.org/lua-filters.html
\end{itemize}

\end{document}
